\section{Data}

One important input for measuring aggregate labor market slack is the Bureau of Labor Statistics' (BLS) Part-Time for Economic Reasons (PTER) indicator, included in the BLS' U-6 measure of labor market underutilization. PTER measures the number of employed part-time workers who would like to work full-time but are not able to due to economic circumstances (e.g., unfavorable business conditions or an inability to find full-time work). A major shortcoming of PTER is that it cannot capture labor market slack among \texti{full-time} workers. We use two panel survey datasets with questions regarding workers' labor market hours constraints to extend measures of labor market slack to incorporate full-time workers who wish to work additional hours. The rich set of variables in these panel datasets also allows us to conduct supplementary analyses on the causes and consequences of such labor market hours constraints.

\subsection{The PSID and the HRS}
The main data source for our analysis is the Panel Study of Income Dynamics (PSID). We also use the Health and Retirement Study (HRS) data to supplement our analysis to cover more recent years. Both the PSID and the HRS are longitudinal household surveys nationally representative for the United States. In addition to rich employment, income, and demographic data, both surveys collect information on labor market hours constraints and home production. The two surveys asked these questions in different periods of time and, as a result, covered different generations and cohorts of the population. Specifically, the PSID collected this information annually from 1968 to 1986, and the sample of the HRS, a biennial survey, covers the period from 1992 to 2012. Another difference between the two surveys is that the PSID does not focus on any particular age group, whereas the HRS respondents are 50 years and older.

Table \ref{Variable} summarizes relevant information collected by the PSID and the HRS. From 1968 to 1987 the PSID asked household heads the following four questions, referencing to the year prior to the survey year:

\begin{enumerate}
\item {``Would you have liked to work more if you could have found more work?''}

\item {``Was there more work available on (any of your jobs) so that you could have worked more if you had wanted to?''}

\item {``Would you have preferred to work less even if you had earned less money?''}

\item {``Could you have worked less if you had wanted to?''}
\end{enumerate}

\noindent We count the head as being upside (downside) constrained if the head reported that he/she wanted to, but was not able to, work more (fewer) hours.  The HRS asks essentially the same questions in all waves. However, unlike the PSID, the HRS questions refer to the time of the survey.\footnote{HRS also collects time use information. However, such information is collected in the ``off-year,'' i.e., one year after the main survey. As a result, unlike the PSID, the time use information and labor market hours constraints information in the HRS does not refer to the same year, limiting the extent to which we can use these data.} One advantage of the HRS data is that in addition to these qualitative questions, the survey also asks the constrained households what their desired number of market work hours were, a question not asked in the PSID.

We use both a direct and an indirect measure of time spent on home production. The direct measure is the annual hours of housework, and the indirect measure is the ratio between expenditures on eating out and total food expenditures (henceforth the food-out ratio). A lower food-out ratio implies more eating and cooking at home, hence a higher level of housework. While both the PSID and the HRS collected housework hours information, the HRS (a biennial survey) data was collected one year off from the reference year of labor hour constraints. As a result, unlike the PSID, the HRS labor hours constraints and housework hours data are referring to the same period.  The food-out ratio can be constructed for both the PSID and the HRS samples and the reference years were the same as the labor hours constraints, allowing us to use the HRS sample, which covers recent years, to supplement and corroborate the results using the PSID data.\footnote{The PSID did not ask food-out expenditure separately in 1968 and skipped the entire food expenditure question in 1973. See Charles, Danziger, Li and Schoeni (2008) for more details about the food expenditure variables in the PSID data. The HRS did not collect food data in 1998.}

We restrict the PSID sample to households whose heads are between 22 and 67 years old and the HRS sample to households whose heads are between 50 and 67 years old. We further restrict the sample to households whose heads were either working or temporarily laid off, remove those with over 4,160 annual labor market work hours or housework hours (over 80 hours per week), and remove those with extremely low or high total family income.

\subsection{Summary Statistics}

Table \ref{Summary} presents summary statistics of key relevant variables in the PSID and HRS pooled cross-section samples. All statistics are estimated using respective PSID and HRS weights and all income and expenditure variables are adjusted using the 1986 dollars. Our estimates indicate that a significant share of workers are bound by market work hours constraints. Specifically, in the PSID sample the upside constraints of labor market hours are binding more frequently than the downside constraints. Nearly 19 percent of our sample observations (household $\times$ year) reported that the household heads were not able to increase work hours when they wanted to (column 1). However, fewer than 6 percent of the sample were not able to decrease hours when they wanted to (column 3). In contrast, the two constraints bind with more similar likelihood in the HRS data (columns 2 and 4). Several factors likely have contributed to the differences between the PSID and the HRS estimates. First, older households on average have more assets that help smooth consumption against income fluctuations, reducing their demand for working longer hours. Second, it is possible that older workers are more likely to prefer working fewer hours because of health-related reasons. Indeed, the PSID households older than 50 had a lower odds (13.3 percent) of being upside constrained households and a higher odds (6.3 percent) of being downside constrained.  Third, part of the remaining difference may be related to the structural and institutional changes in the labor markets---the PSID data cover from late-1960s to mid-1980s, whereas the HRS data cover from early 1990s to most recent years.

The PSID also asked whether a worker, either salaried or paid hourly, could get extra pay for extra work.  There were 47 percent of household heads in our sample who reported being able to receive extra pay for extra work.  Presumably, the upside constraints were more relevant and applicable for such workers. Indeed, 64 percent of the upside-constrained workers reported able to receive extra pay for extra work. Among such workers, 25.5 percent reported being upside constrained, comparing with 12.8 percent among the workers not to receive extra pay for extra work.  In a most conservative estimates that assumes the upside constraints applying to only workers eligible for extra pay, about 12 percent of workers in our sample were constrained.

Because both surveys track the same households over time, we can compute the share of household heads that had ever been constrained during the sample period. For the PSID sample, 52 percent of sample households had been upside constrained, more
than doubling the share of ever-downside-constrained households (25 percent). Only 36 percent of the households were never constrained during the sample period. For the HRS sample, 23 percent of household heads had been upside constrained, 17 percent downside constrained, and 63 percent never reported being constrained in either direction. In addition, the HRS data indicate that among the upside constrained household heads, the hours they worked were on average 450 hours lower than their desired hours.  Among the downside constrained household heads, the hours they worked were on average 730 hours higher than their desired hours. Relative to a full-time worker who work 2000 hours in a year, these estimates imply a 23 percent underwork and a 36 percent overwork margin.

Regarding demographic differences among workers under various constraint status, we find that, on the one hand, the downside constrained household heads were more similar to the unconstrained, and the similarity is more pronounced in the PSID sample. On the other hand, the upside constrained household heads were significantly different from the downside constrained and unconstrained heads. In particular, the upside constrained heads tended to be younger, more likely to be nonwhite, having lower educational attainments. The HRS sample upside constrained were also more likely to be widowed or divorced. In addition, the downside constrained households earned a substantially lower family income, and their heads worked fewer hours comparing with the unconstrained workers. In contrast, the downside constrained workers had a higher average hours worked, reassuring the relevance of the self-reported constraint status. Furthermore, we note that the mean hours worked of the upside constrained were around 2,000 hours per year, suggesting that the upside constraints were not only applicable to part-time workers. Indeed, in the PSID sample, 28 percent of the upside-constrained worked part-time, comparing with 19 percent among the unconstrained.  In the HRS sample, the unconstrained had a greater share of people working part-time than the constrained.

Finally, looking at the housework hour and food-out ratio variables. Because married and unmarried households have different home production preferences, the estimates are presented separately. Indeed, in the PSID data across constraint status, married households tended to have higher housework hours and lower food-out ratios. Our estimates indicate that, in both the PSID and the HRS data, the upside constrained households had greater housework hours and lower food-out ratios, a contrast that holds among both married and unmarried households. The downside constrained households had similar or slightly lower housework hours and slightly higher food-out ratios.

