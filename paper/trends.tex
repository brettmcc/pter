\section{Characterizing Working Hours Constraints}

\subsection{Occupations and Industries of the Constrained}
Working for what types of jobs and in which industries would make people more likely to have hours constraints?  Regarding occupations, as shown in table \ref{OCCIND}, the four occupations with highest odds of upside constraints include nonfarm laborers, operatives, transportation operators, and farm laborers, each with over or near 25 percent of workers being constrained.  The four occupations with lowest odds of upside constraints include sales, professionals, managers, and farmers.  The odds differences across occupations were larger. The average of the top four occupations was 28.6 percent, more than three times higher than the average of the bottom four occupations (8.9 percent).  Interestingly, the odds of upside and downside constraints were not inversely correlated across occupations.\footnote{The correlation coefficient is insignificant and small.}  For example, operatives ranked second for both constraints.

Turning to the odds of constraints in different industries, we find that the construction and entertainment industries had the highest odds of upside constraints (both above 25 percent), whereas the agriculture and finance industries have the lowest odds (near 10 percent).  Also, we note that the cross-industry dispersion of the upside-constraint odds is nearly 40 percent smaller than the cross-occupation dispersion.

\subsection{Trends of Working Hours Constraints}

The PSID estimates indicate that a large share of prime-age household heads were unable to work up to his/her desired number of hours in the labor market during the period from late 1960s to the middle of the 1980s. Among the constrained workers, over 60 percent can benefit financially and earn extra income should they be able to work more hours. Moreover, over 70 percent of the constrained worked full time. The significant prevalence of such labor hours constraints prompts the question whether the standard metrics such as share of people involuntarily working part time reveal the full picture of such frictions in the labor market and whether the survey-based measures of labor hours constraints may shed light on the state of the labor market over business cycles.

The most widely used indicator for tracking involuntary part-time workers is complied by the Bureau of Labor Statistics (BLS) using the Current Population Survey (CPS) data, which is often referred to as the share of working part-time for economic reasons (PTER). To introduce the PSID and the HRS data as useful additional data sources of studying labor market constraints, we show that the similarly defined series these two surveys share similar trends as the BLS PTER series.   Specifically, we show that the shares of involuntary part-time workers (those who were upside constrained and working fewer than 35 hours per week) measured in the PSID and HRS resemble the PTER rates estimated using the CPS data.  As shown in figure 1, during the overlapping sample period of 1967 to 1986, the CPS PTER series and the share of involuntary part-time workers in the PSID have similar levels and fluctuations, with the correlation of the two series being about 0.7, and about 0.6 for the detrended series. Both the PTER and the PSID series are highly countercyclical, with more people involuntarily working part-time or unable to work up to the ideal number of hours during recessions. Interestingly, peaks in the PSID series appear to overlap with NBER recession dates more closely than the PTER series.  Likewise, as shown in figure 2 the CPS PTER estimates of people older than 55 (a subpopulation comparable to the HRS sample) also share a trend similar to the share of upside constrained in the HRS, with both showing a substantial increase during the Great Recession.\footnote{The two series have a correlation coefficient of near 0.85 (0.74 for the detrended series).}  In addition, the CPS PTER estimates of the entire population (the black series) shared a similar trend as the estimates of those aged 55 and above.   All told, the PSID and HRS appear to have measured involuntary part-time work in a way consistent with the CPS PTER.

Because only a fraction of those upside constrained were part-time workers, the PTER series does not reveal the full extent to which the labor force was under working hours constraints. Indeed, as shown in figure 3, the overall prevalence of upside hour-constraints seen in both the PSID and HRS is much higher than the PTER rates, and such a gap is quite stable over the past several decades.  Second, despite the sizable gaps in their levels, the PSID and HRS work-hour-constraint estimates are highly correlated with the CPS PTER series over the respective overlapping period.

\subsection{Persistence of Working Hours Constraints}
To the best of our knowledge, most of the existing work pertinent to the transition dynamics of market work hours constraints uses the CPS data and focuses on involuntary part-time workers.  Because the CPS is a relatively short panel, the analysis is largely limited to the dynamics within a year.  As a result, relatively little is known regarding the reoccurrence and persistence of such constraints over a longer period, in particular among those who work full time. which can be an important element to further understanding of the nature and dynamics of these constraints. We present in table \ref{Markov} some simple statistics of the transition dynamics between years $t$ and $t+1$, and how the dynamics vary by part-time status. Because the upside constraints are much more prevalent and common than the downside constraints and have more direct welfare and monetary implications, our analysis will focus on such constraints.

Panel A shows the raw transition matrix of three states--upside constrained, unconstrained, and downside constrained--for all PSID household heads in our sample that can be linked between two years. We note that the upside constraints are rather persistent, with slightly fewer than half (48.2 percent) of the constrained workers remaining under such constraints one year later. Downside constraints, by contrast, are less persistent, with about one third of the constrained workers remaining so-constrained one year later. In addition, each year, 11.1 percent and 4.7 percent the unconstrained workers became upside and downside constrained, respectively.

Panel B illustrates how the transitions in and out of the upside constraints vary with the part-time status and presents four 2$\times$2 transition matrices that correspond to working full-time in both years, part-time in both years, changing from full-time to part-time, and from part-time to full-time, respectively.  The panel reveals several patterns: first, the upside constraints were fairly persistent across all four groups of workers, with the remaining constrained likelihood between 42 to 56 percent. Second, part-time workers appear to have a higher chance of remaining constrained, but those switch to working full-time in year $t+1$ had a better chance of escaping the constraints. Third, the unconstrained employees who changed from working full-time to working part-time had the highest chance of becoming constrained (17.7 percent), nearly twice as high as the odds of those who continued working full-time (9.8 percent). On balance, the statistics in panel B are consistent with the notion that many of those working part-time did so involuntarily and were vulnerable to such constraints.

Panel C takes a different perspective and presents the year $t+1$ status of those who were constrained in year $t$ by their part-time status in year $t$. Among the constrained full-time workers, about 85 percent stayed working full-time and 15 percent changed to working part-time. The full-time--part-time transition rate among the constrained appears to be higher than this rate for the entire labor force (see, for example, Borowczyk-Martins and Lal\'{e} 2019).\footnote{Another factor accounting for difference between our statistics and those reported in Borowczyk-Martins and Lal\'{e} (2019) is that the latter includes transitions into unemployment, other employment, and out of labor force, whereas our sample includes only those working positive hours in both year $t$ and year $t+1$.} Regardless whether changing to working part-time, roughly half of the workers remained constrained the next year. For part-time workers, about 55 percent of those constrained in year-$t$ worked full-time the next year.  Interestingly, 56 percent of those working full-time continued to report having working hours constraints, comparing with 42 percent of those remaining working part-time. The difference suggests that changing to working full-time was due to hours constraints. However, working full-time alone, for many workers, did not completely relax the constraints.

Finally, we note that the results presented in table \ref{Markov} are little changed when we restrict the sample to those who would receive additional pay if they worked extra hours.

\subsection{Factors Accounting for Transitions in and out of the Constraint}

We now estimate a econometric model to quantify the extent to which various demographic, labor market, and family income factors are associated with transition into and out of upside hours constraints.  We underscore that this is not a structural model that intends to uncover causal relationship.  Rather, we are interested in the degree to which such transitions can be accounted for by worker characteristics observed by econometricians. Specifically, we estimate the following logistic model for becoming upside constrained:

\begin{eqnarray}
Enter_{t, t+1}^i & = & \beta_0 + \beta_1 Age^i + \beta_2 Black^i + \beta_3 Edu^i + \beta_4 Lifecycle^i + \beta_5 \Delta Income_{t, t+1}^i \\
                 &   & + \beta_6 PT\sim FT_{t, t+1}^i + \beta_7 Year_t. \nonumber \label{logitbc} 
\end{eqnarray}

\noindent where $Enter_{t, t+1}^i$ is a indicator that is equal to one if worker $i$ became upside constrained in $t+1$, $Age$ is a third-order polynomial of worker's age, $Black$ a race dummy, and $Edu$ a vector of dummies indicating worker's educational attainment.  In addition $Lifecycle$ is a vector of lifecycle events including homeowner status, marital status, and number of children changes.  $\Delta Income$ includes log differences of family income and wage rate between $t$ and $t+$, and $PT-FT$ is a vector of full-time, part-time work status changes, with working full-time in both years being the omitted group.  Finally, we control for year fixed effects, $Year_t$.  The model is estimated using the sample of workers unconstrained in year $t$.  The estimated odds ratios are reported in column 1 of table \ref{TransitionReg}.  Odds ratios of continuous variables are evaluated for a one-standard deviation change.

Our results reveal that black workers on average are 72 percent more likely to become constrained than otherwise comparable white workers in a given year, and that the likelihood of becoming constrained substantially diminishes with education levels.  Regarding lifecycle events, workers remained married had a higher odds of becoming constrained relative to the workers remained single (the omitted group).  Moreover, having an additional child implies a 17 percent higher chance of becoming constrained, likely reflecting increased need of family spending. Indeed, a one-standard deviation increase in log of family income led to a nearly 45 percent reduction of the likelihood of being constrained, consistent with a positive income effect on labor supply.  Moreover, higher wage rates were associated with higher odds of becoming constrained, consistent with a positive substitution effect. Regarding part-time status changes, relatively to those who worked full-time in both years (the omitted group), other workers had a higher chance of becoming constrained, suggesting that such hours constraints were more likely to be binding for part-time workers.  Specifically, people who worked part-time in year $t+1$ had a 60 to 70 percent higher odds than the omitted group, and the year-$t$ part-time workers who became full-time in year $t+1$ has a 20 percent higher odds.  Moreover, we estimate the model using such unconstrained workers who would get extra pay for working extra hours (column 2). Most of the results are qualitatively the same as those in column (1), with the race estimates and the gradient of educational attainment estimates being somewhat smaller. One notable feature of the results presented is the low levels of the R-squared, 0.066 for the entire sample of year-$t$ unconstrained and 0.038 for the subsample of extra-pay eligible workers, which highlight that transitions into such constraints are only accounted for to a limited extent by observable factors.

We next study the transitions out of upside hours constraints, using a model similar to eq.(\ref{logitbc}). We add two additional variables, changed jobs and got additional jobs, to test if these were effective means of escaping such constraints.  We find that, among all workers reported being constrained in year $t$, black workers and workers with lower levels of education were less likely to escape the constraint (column 3).  Workers who remained married in both years also had a lower likelihood of escaping relative to those who stayed single. However, changes of number of children were not associated with the odds of becoming unconstrained. Consistent with a positive income effect and negative substitution effect on labor supply, higher family income and lower wages were associated with greater likelihood of no longer being constrained. In addition, workers working part-time in both years were less likely to become unconstrained than those working full-time in both years, though workers with full-time to part-time or part-time to full-time transitions appear to have similar odds as the omitted group.  Finally, our estimates suggest that constrained workers who changed job between year $t$ and $t+1$ had a greater odds of becoming unconstrained, but getting additional jobs did not change the likelihood.  The results estimated with the extra pay-eligible subsample (column 4) are largely the same, with the exception that race no longer predicts constraint status changes.  Similar to columns 1 and 2, the estimated values of R-squared were quite low.       