\documentclass[12pt]{article}
\usepackage{amsmath}
%\usepackage[xdvi]{graphics,color}
%\usepackage[xdvi]{graphicx}
\usepackage{graphicx}
\usepackage{epstopdf}
\usepackage{threeparttable}
%\usepackage{harvard,setspace,epsfig}
\usepackage{soul}
\usepackage{epsfig}
\usepackage{natbib}
\usepackage[justification=centering]{caption}
\usepackage{setspace, lscape}

\usepackage{amsmath,amssymb,array}
\usepackage{float,afterpage,setspace}
\usepackage{multicol,multirow,dcolumn,longtable,lscape,threeparttable}
\usepackage{color,colortbl}

\setlength{\textwidth}{17.1cm} \setlength{\textheight}{22.5cm}
\setlength{\headheight}{-1cm} \setlength{\footskip}{1cm}
\setlength{\oddsidemargin}{-0.5cm}
\setlength{\evensidemargin}{-0.5cm}

\begin{document}
\baselineskip 24pt
\newtheorem{proposition}{Proposition}
\newtheorem{lemma}{Lemma}
\newtheorem{assumption}{Technical Assumption}
\def\citealt#1{\citename{#1}, \citeyear*{#1}}

\section{Model for Individual}

Consider a stylistic, simple model of one individual optimizing market and housework labor supplies.  She chooses working $L$ hours in the labor market, earning a unit of goods per hour, and $H$ hours on housework.  Assuming her consumption is produced by combining housework and goods purchased with wages earned in the labor market with a Cobb-Douglas technology $C = L^\alpha H^{(1-\alpha)}$. Her utility function is defined as $U = log(C) + log(1 - L - H)$, where $1-L-H$ indicates her leisure time net of market and housework. The optimality condition $\displaystyle\frac{dU}{dL} = \frac{dU}{dL} = 0$ implies the well known result $\displaystyle \frac{L^*}{N^*} = \frac{\alpha}{1-\alpha}$. Solving for optimal market and housework hours, we have $\displaystyle L^* = \frac{\alpha}{2}$ and $\displaystyle H^* = \frac{1-\alpha}{2}$.  The optimization has an interior solution if $\underline{L} \leq L^* \leq \overline{L}$, where $\underline{L}$ and $\overline{L}$ are the lower and upper bounds of labor market hours, respectively. If the individual has a binding market hour constraint, $\overline{L} < L^*$.  At $\overline{L}$, $\displaystyle \frac{dU}{dL} > 0$, i.e., the marginal value of leisure is too high relative to the marginal value of consumption. Accordingly, she will increase $H$ above $H^*$.  Specifically,  $\displaystyle\widehat{H}^* = \frac{(1-\alpha)(1 - \overline{L})}{2-\alpha}$, where the optimal housework hours under binding market hour constraints decreases with $\overline{L}$.

\section{Model for Couple}

Now consider a model for a couple with spouse A and spouse B. The optimization program becomes 

\begin{equation}
\max_{L_A, L_B, H_A, H_B} log(C) + log(1 - L_A - H_A) + log(1 - L_B - H_B), \nonumber
\end{equation}

\begin{math}
\text{where} \;\;\; C = (L_A + mL_B)^\alpha(H_A^\beta H_B^{1-\beta})^{1-\alpha}.
\end{math}

The couple consumes $C$, which is a bundle of purchased goods and housework input. Note that $m$ denotes the relative market labor productivity (wages) between spouses. The technology of producing $C$ reflects the assumption that the spouses' market labor income is perfect substitutes but their housework input is not. In addition, the couple values about each individual's leisure equally. The first order conditions are

\begin{equation}
\frac{\alpha}{L_1 + mL_2} = \frac{1}{1 - L_1 - H_1} = \frac{(1-\alpha)\beta}{H_1}, \;\;\; \text{and}  \nonumber
\end{equation}

\begin{equation}
\frac{\alpha m}{\L_1 + mL_2} = \frac{1}{1 - L_2 - H_2} = \frac{(1-\alpha)(1-\beta)}{H_2}, \nonumber
\end{equation}

which imply 

\begin{equation}
\frac{1}{m} = \frac{1 - L_2 - H_2}{1 - L_1 - H_1} = \frac{\beta H_2}{(1-\beta) H_1}. 
\end{equation}


Assuming spouse $B$ has a lower market wage ($m < 1$) and a higher equilibrium share of housework ($\beta < \frac{1}{2}$), we have $L_1 > L_2$ and $H_1 < H_2$.  Note that, for sufficiently small $m$, equation (1) implies $L_2 = 0$ because of nonnegative market labor hours. The time allocation is consistent with the patterns observed for the majority couples in our sample, where the household heads earned a higher wage and worked more hours than the spouses in the labor market. Some spouses worked zero hours in the labor market. In addition, most heads worked positive but fewer hours than spouses on housework. 



\end{document} 