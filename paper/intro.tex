\section{Introduction}
A great deal of attention from policy makers and researchers has focused on the elevated level of workers who are part-time for economic reasons (PTER) since the Great Recession. While PTER is an important measure produced by the Bureau of Labor Statistics to measure labor market slack in the economy, we find that it significantly understates it. Using two nationally representative household surveys, we find that many full-time workers cannot work as many hours as they would prefer.

The survey data we use are the Panel Study of Income Dynamics (PSID) and the Health and Retirement Study (HRS).  Each survey asked respondents, “Would you have liked to work more if you could have found more work?” and, “Was there more work available on (any of your jobs) so that you could have worked more if you had wanted to?” We count a worker as facing work-hour constraints if he responded that he wanted to, but was not able to, work more hours. Separately, both surveys also collected data on how many hours an individual worked in a typical week, which let us identify the part-time workers and full-time workers.

We find that the share of part-time workers who were facing work-hour constraints match with the CPS PTER series very closely , indicating that the survey questions capture the PTER concepts well.  However, the share of all workers (part-time and full-time) who are work-hour constrained, while sharing a similar trend, is much higher than the CPS PTER rate, suggesting labor market slack is much more pervasive than the PTER rate alone implies.  Notably, the margin widened during the Great Recession and its aftermath (see figure below).  We then study the transition dynamics of such work-hour constraints and find that their persistence is quite high.  Fifty percent the constrained workers remain constrained the next year and, interestingly, the ratio is not sensitive to the duration of the experienced constraints.

We also explore the factors that drive workers into the constrained situations and how constrained workers deal with such scenarios.  We find that having an additional child in the family is not predictive for becoming constrained (holding earnings constant), while reductions in family income or increases in wage rates are associated with a greater likelihood of being work-hour constrained.  We also find that constrained workers increase home production, measured using housework hours and food-at-home share in total food expenses, to equalize the marginal utility of consumption and leisure in the constrained scenario.