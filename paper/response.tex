\section{Households Responses to Labor Hours Constraints}

\subsection{Conceptual Framework}

How do workers and their families deal with such labor hours constraints?  When worker want to work more hours than they are able to, the standard labor supply model implies that their marginal utility of leisure is not as large as the marginal utility associated with the additional income earned from working the marginal hour. To equalize these margins and achieve a second-best optimal, workers may increase home production or increase labor supply of other household members should it be available.\footnote{In addition to these two aspects, constrained workers may also reduce consumption, but saving, or increase borrowing.  We do not study these responses in this paper.} [I will add a stylized model that gives us increases in house work hours and spouse labor supply (for married households).  The model will be similar to the one in the earlier version but will consider spouse labor supply.  However, the model will not take into account intra-household bargaining.]

\subsection{Empirical Results}

\subsubsection{Single Households}

We first study how single workers respond to labor hours constraints. Specifically, we estimate the following model to test if single workers increase housework hours, $HWHead$, in response to upside labor hours constraints, $UC$, and reduce housework hours in response to downside constraints, $UC$.

\begin{equation} \label{housework}
HWHead = \alpha + \beta_u UC + \beta_d DC + \gamma Z + \theta Income + \phi Part-time + \tau Year,
\end{equation}

where demographic vector $Z$ includes an age polynomial and race and education dummies; $Income$ represents a vector of family income decile dummies, $Part-time$ indicates whether the head worked part-time in a given year; and the model also controls for year fixed effects.  We first estimate a similar linear regression, the results of which are reported in column 1 of table \ref{Single}. Our estimates indicate that single workers put, on average, about 47 more hours in housework during the year the worker was upside constrained---8.5\% higher than the unconstrained. However, housework hours did not appear to change when workers were under downside constraints. In addition, the estimated control variable coefficients suggest that households whose heads were black, had lower educational attainments, were single female, or were working part-time tend to have more housework hours.  Moreover, more children in the household tend to call for more housework. Note that the distribution of housework hours is non-normal and censored at zero.  Accordingly, we estimate a more robust specification, the tobit model.  The results, shown in column 2, are largely the same as in the linear model. The higher housework hours of constrained workers may reflect unobserved, persistent factors, such as household-specific preferences, instead of the labor hours constraints.  To quantify this potential effect, we estimate a longitudinal model, controlling for individual fixed effects.  The estimated increase of housework hours, 26 hours, is appreciably smaller than in the linear and tobit model, but remain sizeable and statistically significant. In addition to unobservable factors, the smaller estimated coefficient of $UC$ may also be attributable to the upside hours constraints being rather persistent. Finally, In all specification, downside hours constraints did not appear to have a significant bearing on housework hours.

We then reestimated these models using the subsample of workers who could earn extra money if working extra hours. Arguably, these were the workers with most tangible financial interests of working longer hours, therefore representing the most relevant hours constraints. The results, presented in columns 4-6, are qualitatively the same as the results estimated using the entire sample, with somewhat larger point estimates.  For example, the linear and tobit models show upside constrained workers had over 55 more housework hours.

\subsubsection{Married Households}
Turning to married households, these households had three ways to respond to heads' labor hours constraints as indicated by the model outlined above. Consider the upside constraints, first, the head could increase his own housework hours; second, the spouse could increase her housework hours; and third, the spouse could increase her market labor supply as she was not under such hours constraints. We test these channels in the framework of equation (\ref{housework}), replacing $HWHead$ with the three margins married households could adjust. In addition, we add spouse educational attainment as an additional control.  The results are presented in table \ref{Married}. 

The PSID define the male of a couple as the head of a married household. First, we note that married household heads' response to upside labor hours constraints (columns 1-2) were more muted than single male workers, and this pattern held in both the tobit (31 versus 47 hours) and longitudinal regressions (10 versus 26 hours). In addition, like single workers, housework hours of married households did not appear to change for downside constraints ($DC$). Turning to wife's housework, as shown in columns 3-4, heads' being unable to work more hours appeared to be associated with lower wife's time input for home production, as the coefficient of $UC$ is negative in both specifications. In contrast, wives of heads with downside hours constraints had more hours of housework than otherwise comparable spouses. There could be two factors accounting for why heads' labor hours constraints did not lead to the same change to wives' housework hours as they did on their own housework hours.  First, wives had a large number of housework hours, around 1,400 hours per year, regardless heads' labor hours constraint status, comparing with about heads' 300 housework hours.  Therefore, the marginal cost of wives' housework could be so high that increasing it even when heads were constrained was not desirable.  Second, if wives chose to increase market labor supply in response to heads' labor hours constraints, that will reduce their available time for housework.

Columns 5-9 explore wives' market labor supply changes, on both the extensive and intensive margins. We first replace the dependent variable of equation (\ref{housework}) with a dummy of whether wife worked in a given year and estimate a logistic model. The estimated odds ratios indicate that wives of upside constrained heads were over 20 \% more likely to work in that year. However, as in many of the previous results, downside constraints did not have a significant bearing on wife work status. In addition, our estimates reveal that wives' propensity of working declined with heads' education and rose with own education. The former likely reflects an income effect of wife labor supply, whereas the latter reveals a substitution effect as more educated women earn higher wages. Also, we note that female labor force participation rate in the earlier part of our PSID sample period remained low. We next study if wives who were not working before would start working when heads were constrained, by estimating the previous model using a subsample of wives not working in year $t-1$.  As shown in column 6, relative to other comparable spouses, wives are more 16 percent likely to enter the labor market during the year when their husbands were not able to work more hours as desired.

On the intensive margin, we find that working wives whose spouses were upside constrained tended to work more hours.  First, as shown in column 7, upside constrained heads' working wives were 8 percent less likely to work part-time, whereas the downside constrained heads' spouses were 20 percent more likely to do so.  Finally, the cross-sectional and longitudinal analysis indicate that wives' working hours are estimated to be 53 and 36 hours longer, respectively, in the year their husbands are upside constrained. The downside constraints were not strongly associated with wives working hours.

To summarize, we find that single workers appeared to increase housework hours significantly when they were not able to work as many hours as desired in the labor market.  For married households, the adjustments on housework were more muted. In contrast, wives tended to increase their market labor supply under such constraints, either by entering the labor force or by working longer hours. Results presented in table \ref{Married} are estimated using the entire sample of married households. Reestimating the models using the subsample of extra-pay eligible workers yields qualitatively the same results.

\subsection{Food-out Ratios}
The PSID analysis, while comprehensive, draws on a sample the last observation of which was collected almost 35 years ago.  We supplement the study with an analysis of the HRS data, which cover the recent year. First, we note that unlike in the PSID sample, we did not find labor supply variations of HRS wives associated with their husbands' hours constraints. The difference likely reflects the sample age differentials.  The youngest household heads in the HRS were 50 years old, and wives' labor supply flexibility is more limited at that age.  Indeed, when we repeat the analysis using the PSID subsample of households aged 50 years and older, we no longer see spouse labor supply changes.
 
We therefore focus on the potential effect of labor hours constraints on housework and home production.  As discussed earlier, the HRS data collect housework hours data but in a year different from labor hours constraints data. As a result, we use food-out ratio, defined as the share of eating expenditures in total food expenditures as a proxy of housework, with lower food-out ratio indicating greater home production. This ratio can be constructed in both the PSID and HRS data, contemporaneous to the labor hours constraints data. Another difference between the two surveys is that the food-out ratios differed by a large margin between single and married households in the PSID data but were close in the HRS data, which is also mainly due to age differences of the two survey. Food-out ratios were similar in the PSID subsample of households older than 50.

We first confirm if the housework hour variations we find in the PSID manifest in food-out ratios. Doing so will further validate using the food-out ratio as a proxy of housework. As shown in column 1-4 of table \ref{HRS}, estimating a tobit specification of equation (\ref{housework}) reveals that, for single and married households, upside hours constraints were associated with food-out ratios 1.5 and 0.9 percentage point (about 6 percent of mean ratio) lower, respectively.  The estimated reduction are greater for extra-pay eligible workers at around 2 percentage points.

Turning to the HRS sample, which we estimate the model pooling single and married households together, constrained households were estimated to have a 2 percentage points lower food-out ratio, or 8 percent of the mean ratio of unconstrained households. Interestingly, the estimate is appreciably smaller among extra-pay eligible households but remains statistically significant.\footnote{The difference relative to the PSID in part reflects variations in how extra-pay eligibility information was collected in both surveys as the labor market evolved over the past several decades.}
